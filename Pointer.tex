1. 什么是指针?(What is a pointer?)
>> 指针就是用来存储 内存地址 的变量 (A pointer is a variable that holds a memory address)

例如:
我们在C++中执行:  int a = 5;
那么这个 a 变量就会被 compiler 分配一个内存地址(假设分配到了 102)
-----------------
|Address| Value |
-----------------
| 100   |       |
-----------------
| 101   |       |
-----------------
| 102(a)|   5   |
-----------------
| 103   |       |
-----------------
| 104   |       |
-----------------
| 105   |       |
-----------------
| 106   |       |
-----------------

接着,在之后的代码中如果想要调用 变量 a,那么调用过程为:
先找到 RAM[Address] --> RAM[102] --> 5 (Value)


==========================分界==============================
接下来我们在C++中创建一个pointer(指针):
pointer b = 100;      // 100 是 Address
一般来说很少有 上面的写法 (直接指定一个pointer b 所指向的 内存地址的数字编号 100)

我们通常会 通过另一个变量的地址 来为pointer b 来赋值:
int b = &a;       // 这里  &a 就表示 变量 a 的内存地址
与此同时,b 也需要有自己的 内存地址 RAM[Address]。我们假设 b 自己的 内存地址为 104:
-----------------
|Address| Value |
-----------------
| 100   |       |
-----------------
| 101   |       |
-----------------
| 102(a)|   5   |
-----------------
| 103   |       |
-----------------
| 104(b)|  102  |
-----------------
| 105   |       |
-----------------
| 106   |       |
-----------------
这种 用 变量 a 的地址 来为 指针 b 赋值的方法叫做 “Direct Addressing”。


==========================分界==============================
在有了上面的 变量 a 和 指针 b 之后,我们想要查看 指针 b 所指向的 地址(102)该怎么办?
int c = *b;       // 这里的 * 号不是 ‘乘法’,而是:'get the value at somewhere'。
也就是获得 指针 b 所指向的 值(102)。然后将这个值(102) 赋值给 变量 c。与此同时,也给 变量 c 分配一个 内存空间(100)
这种 用 指针 b 所指向的值 来为 变量 a 赋值的方法叫做 “Indirect Addressing”。 从而 a 的值 被间接地 赋予 c。
-----------------
|Address| Value |
-----------------
| 100(c)|   5   |
-----------------
| 101   |       |
-----------------
| 102(a)|   5   |
-----------------
| 103   |       |
-----------------
| 104(b)|  102  |
-----------------
| 105   |       |
-----------------
| 106   |       |
-----------------


